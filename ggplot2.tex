\PassOptionsToPackage{unicode=true}{hyperref} % options for packages loaded elsewhere
\PassOptionsToPackage{hyphens}{url}
%
\documentclass[
]{article}
\usepackage{lmodern}
\usepackage{amssymb,amsmath}
\usepackage{ifxetex,ifluatex}
\ifnum 0\ifxetex 1\fi\ifluatex 1\fi=0 % if pdftex
  \usepackage[T1]{fontenc}
  \usepackage[utf8]{inputenc}
  \usepackage{textcomp} % provides euro and other symbols
\else % if luatex or xelatex
  \usepackage{unicode-math}
  \defaultfontfeatures{Scale=MatchLowercase}
  \defaultfontfeatures[\rmfamily]{Ligatures=TeX,Scale=1}
  \setmainfont[]{Verdana}
\fi
% use upquote if available, for straight quotes in verbatim environments
\IfFileExists{upquote.sty}{\usepackage{upquote}}{}
\IfFileExists{microtype.sty}{% use microtype if available
  \usepackage[]{microtype}
  \UseMicrotypeSet[protrusion]{basicmath} % disable protrusion for tt fonts
}{}
\makeatletter
\@ifundefined{KOMAClassName}{% if non-KOMA class
  \IfFileExists{parskip.sty}{%
    \usepackage{parskip}
  }{% else
    \setlength{\parindent}{0pt}
    \setlength{\parskip}{6pt plus 2pt minus 1pt}}
}{% if KOMA class
  \KOMAoptions{parskip=half}}
\makeatother
\usepackage{xcolor}
\IfFileExists{xurl.sty}{\usepackage{xurl}}{} % add URL line breaks if available
\IfFileExists{bookmark.sty}{\usepackage{bookmark}}{\usepackage{hyperref}}
\hypersetup{
  pdftitle={ggplot2},
  pdfauthor={Fabian Koch},
  pdfborder={0 0 0},
  breaklinks=true}
\urlstyle{same}  % don't use monospace font for urls
\usepackage[margin=1in]{geometry}
\usepackage{color}
\usepackage{fancyvrb}
\newcommand{\VerbBar}{|}
\newcommand{\VERB}{\Verb[commandchars=\\\{\}]}
\DefineVerbatimEnvironment{Highlighting}{Verbatim}{commandchars=\\\{\}}
% Add ',fontsize=\small' for more characters per line
\usepackage{framed}
\definecolor{shadecolor}{RGB}{248,248,248}
\newenvironment{Shaded}{\begin{snugshade}}{\end{snugshade}}
\newcommand{\AlertTok}[1]{\textcolor[rgb]{0.94,0.16,0.16}{#1}}
\newcommand{\AnnotationTok}[1]{\textcolor[rgb]{0.56,0.35,0.01}{\textbf{\textit{#1}}}}
\newcommand{\AttributeTok}[1]{\textcolor[rgb]{0.77,0.63,0.00}{#1}}
\newcommand{\BaseNTok}[1]{\textcolor[rgb]{0.00,0.00,0.81}{#1}}
\newcommand{\BuiltInTok}[1]{#1}
\newcommand{\CharTok}[1]{\textcolor[rgb]{0.31,0.60,0.02}{#1}}
\newcommand{\CommentTok}[1]{\textcolor[rgb]{0.56,0.35,0.01}{\textit{#1}}}
\newcommand{\CommentVarTok}[1]{\textcolor[rgb]{0.56,0.35,0.01}{\textbf{\textit{#1}}}}
\newcommand{\ConstantTok}[1]{\textcolor[rgb]{0.00,0.00,0.00}{#1}}
\newcommand{\ControlFlowTok}[1]{\textcolor[rgb]{0.13,0.29,0.53}{\textbf{#1}}}
\newcommand{\DataTypeTok}[1]{\textcolor[rgb]{0.13,0.29,0.53}{#1}}
\newcommand{\DecValTok}[1]{\textcolor[rgb]{0.00,0.00,0.81}{#1}}
\newcommand{\DocumentationTok}[1]{\textcolor[rgb]{0.56,0.35,0.01}{\textbf{\textit{#1}}}}
\newcommand{\ErrorTok}[1]{\textcolor[rgb]{0.64,0.00,0.00}{\textbf{#1}}}
\newcommand{\ExtensionTok}[1]{#1}
\newcommand{\FloatTok}[1]{\textcolor[rgb]{0.00,0.00,0.81}{#1}}
\newcommand{\FunctionTok}[1]{\textcolor[rgb]{0.00,0.00,0.00}{#1}}
\newcommand{\ImportTok}[1]{#1}
\newcommand{\InformationTok}[1]{\textcolor[rgb]{0.56,0.35,0.01}{\textbf{\textit{#1}}}}
\newcommand{\KeywordTok}[1]{\textcolor[rgb]{0.13,0.29,0.53}{\textbf{#1}}}
\newcommand{\NormalTok}[1]{#1}
\newcommand{\OperatorTok}[1]{\textcolor[rgb]{0.81,0.36,0.00}{\textbf{#1}}}
\newcommand{\OtherTok}[1]{\textcolor[rgb]{0.56,0.35,0.01}{#1}}
\newcommand{\PreprocessorTok}[1]{\textcolor[rgb]{0.56,0.35,0.01}{\textit{#1}}}
\newcommand{\RegionMarkerTok}[1]{#1}
\newcommand{\SpecialCharTok}[1]{\textcolor[rgb]{0.00,0.00,0.00}{#1}}
\newcommand{\SpecialStringTok}[1]{\textcolor[rgb]{0.31,0.60,0.02}{#1}}
\newcommand{\StringTok}[1]{\textcolor[rgb]{0.31,0.60,0.02}{#1}}
\newcommand{\VariableTok}[1]{\textcolor[rgb]{0.00,0.00,0.00}{#1}}
\newcommand{\VerbatimStringTok}[1]{\textcolor[rgb]{0.31,0.60,0.02}{#1}}
\newcommand{\WarningTok}[1]{\textcolor[rgb]{0.56,0.35,0.01}{\textbf{\textit{#1}}}}
\usepackage{graphicx,grffile}
\makeatletter
\def\maxwidth{\ifdim\Gin@nat@width>\linewidth\linewidth\else\Gin@nat@width\fi}
\def\maxheight{\ifdim\Gin@nat@height>\textheight\textheight\else\Gin@nat@height\fi}
\makeatother
% Scale images if necessary, so that they will not overflow the page
% margins by default, and it is still possible to overwrite the defaults
% using explicit options in \includegraphics[width, height, ...]{}
\setkeys{Gin}{width=\maxwidth,height=\maxheight,keepaspectratio}
\setlength{\emergencystretch}{3em}  % prevent overfull lines
\providecommand{\tightlist}{%
  \setlength{\itemsep}{0pt}\setlength{\parskip}{0pt}}
\setcounter{secnumdepth}{-2}
% Redefines (sub)paragraphs to behave more like sections
\ifx\paragraph\undefined\else
  \let\oldparagraph\paragraph
  \renewcommand{\paragraph}[1]{\oldparagraph{#1}\mbox{}}
\fi
\ifx\subparagraph\undefined\else
  \let\oldsubparagraph\subparagraph
  \renewcommand{\subparagraph}[1]{\oldsubparagraph{#1}\mbox{}}
\fi

% set default figure placement to htbp
\makeatletter
\def\fps@figure{htbp}
\makeatother

\setlength{\columnsep}{18pt}
\usepackage{multicol}
\newcommand{\hideFromPandoc}[1]{#1}
\hideFromPandoc{ \let\Begin\begin \let\End\end }

\title{ggplot2}
\usepackage{etoolbox}
\makeatletter
\providecommand{\subtitle}[1]{% add subtitle to \maketitle
  \apptocmd{\@title}{\par {\large #1 \par}}{}{}
}
\makeatother
\subtitle{Options, packages and examples for PDF print}
\author{Fabian Koch}
\date{}

\begin{document}
\maketitle

\newpage 
\tableofcontents

\newpage

\hypertarget{data}{%
\section{Data}\label{data}}

\newpage

\hypertarget{informationen}{%
\section{Informationen}\label{informationen}}

\hypertarget{package-infos-repositories-links}{%
\subsection{package Infos, repositories,
links}\label{package-infos-repositories-links}}

\hypertarget{vorlagen}{%
\section{Vorlagen}\label{vorlagen}}

\hypertarget{color-palettes}{%
\subsection{color palettes}\label{color-palettes}}

\begin{Shaded}
\begin{Highlighting}[]
\NormalTok{PAL_Gliederung_Colour <-}\StringTok{ }\KeywordTok{c}\NormalTok{(}\DataTypeTok{SR =} \StringTok{"blue"}\NormalTok{, }\DataTypeTok{UBZTP =} \StringTok{"orange"}\NormalTok{)}
\NormalTok{PAL_Gebiet_fill <-}\StringTok{ }\KeywordTok{c}\NormalTok{(}\StringTok{"yellow3"}\NormalTok{, }\StringTok{"black"}\NormalTok{, }\StringTok{"grey"}\NormalTok{, }\StringTok{"maroon3"}\NormalTok{) }
\NormalTok{PAL_pal9GnPu <-}\StringTok{ }\KeywordTok{c}\NormalTok{(}\StringTok{"#762a83"}\NormalTok{, }\StringTok{"#9970ab"}\NormalTok{, }\StringTok{"#c2a5cf"}\NormalTok{, }\StringTok{"#e7d4e8"}\NormalTok{, }\StringTok{"#f7f7f7"}\NormalTok{, }\StringTok{"#d9f0d3"}\NormalTok{, }\StringTok{"#a6dba0"}\NormalTok{, }\StringTok{"#5aae61"}\NormalTok{, }\StringTok{"#1b7837"}\NormalTok{)}
\NormalTok{PAL_virpal <-}\StringTok{ }\NormalTok{viridisLite}\OperatorTok{::}\KeywordTok{viridis}\NormalTok{(}\DecValTok{6}\NormalTok{)}
\NormalTok{PAL_col6qual <-}\StringTok{ }\KeywordTok{c}\NormalTok{(}\StringTok{"#66c2a5"}\NormalTok{,}\StringTok{"#fc8d62"}\NormalTok{,}\StringTok{"#8da0cb"}\NormalTok{,}\StringTok{"#e78ac3"}\NormalTok{,}\StringTok{"#a6d854"}\NormalTok{,}\StringTok{"#ffd92f"}\NormalTok{)}
\NormalTok{PAL_div_lowcontrast <-}\StringTok{ }\KeywordTok{c}\NormalTok{(}\StringTok{"#fc8d62"}\NormalTok{,}\StringTok{"#e78ac3"}\NormalTok{,}\StringTok{"#66c2a5"}\NormalTok{, }\StringTok{"#8da0cb"}\NormalTok{,}\StringTok{"#a6d854"}\NormalTok{,}\StringTok{"#ffd92f"}\NormalTok{,}\StringTok{"#e5c494"}\NormalTok{)}
\end{Highlighting}
\end{Shaded}

\hypertarget{themes}{%
\subsection{Themes}\label{themes}}

\href{https://ggplot2.tidyverse.org/reference/theme.html}{Refrence
Manual}

Hier werden themes für unterschiedliche Plot Typen und Größen
festgelegt.

Vorschlag für eine Syntax:\\
theme\_package (ggplot, plotly etc.)\_Plottyp(bspw. map oder
barchart)\_Format(print oder HTML)\_Seitenformat (DINA4 oder A5
etc.)\_SpaltenLayout (1, 2 oder 3-spaltig)

Beispiel für ein Theme für ein ggplot2 plot, Typ Karte, für
Printfassungen im A4 Format, 1-spaltig:\\
theme\_ggplot2\_map\_print\_A4\_1c

/newpage

\hypertarget{snippets}{%
\subsection{Snippets}\label{snippets}}

\hypertarget{beschriftungen}{%
\subsubsection{Beschriftungen}\label{beschriftungen}}

\newpage

\hypertarget{plots}{%
\subsubsection{Plots}\label{plots}}

\hypertarget{karte}{%
\paragraph{Karte}\label{karte}}

\begin{Shaded}
\begin{Highlighting}[]
\NormalTok{mapData <-}\StringTok{ }\NormalTok{data_WorldGeom }\OperatorTok\StringTok{ }
\StringTok{  }\KeywordTok{select}\NormalTok{(}
\NormalTok{    name,}
\NormalTok{    continent,}
\NormalTok{    pop_est,}
\NormalTok{    income_grp,}
\NormalTok{    geometry) }\OperatorTok\StringTok{ }
\StringTok{  }\KeywordTok{filter}\NormalTok{(continent }\OperatorTok{==}\StringTok{ "Asia"}\NormalTok{) }\OperatorTok\StringTok{ }
\StringTok{  }\KeywordTok{mutate}\NormalTok{(}
    \CommentTok{# Vereinigung der 5 Kategorien zu 3}
    \DataTypeTok{income_grp =}\NormalTok{ forcats}\OperatorTok{::}\KeywordTok{fct_collapse}\NormalTok{(income_grp,}
      \DataTypeTok{Hoch =} \KeywordTok{c}\NormalTok{(}
        \StringTok{"1. High income: OECD"}\NormalTok{, }
        \StringTok{"2. High income: nonOECD"}\NormalTok{),}
      \DataTypeTok{Mittel =} \KeywordTok{c}\NormalTok{(}
        \StringTok{"3. Upper middle income"}\NormalTok{, }
        \StringTok{"4. Lower middle income"}\NormalTok{),}
      \DataTypeTok{Niedrig =} \KeywordTok{c}\NormalTok{(}
        \StringTok{"5. Low income"}\NormalTok{)))}
  
\NormalTok{mapPlot <-}\StringTok{ }\KeywordTok{ggplot}\NormalTok{(mapData) }\OperatorTok{+}
\StringTok{    }\CommentTok{# da das data.frame eine geometry Spalte besitzt, kommt geom_sf ohne x und y bzw. Rechts- und Hochwerte aus}
\StringTok{    }\CommentTok{# data.frames mit Rechts- und Hochwerten können über sf::st_as_sf in dieses Format konvertiert werden}
\StringTok{    }\CommentTok{# https://www.rdocumentation.org/packages/sf/versions/0.9-7/topics/st_as_sf}
\StringTok{    }\CommentTok{# https://r-spatial.github.io/sf/reference/st_as_sf.html}
\StringTok{    }\KeywordTok{geom_sf}\NormalTok{(}
      \DataTypeTok{data =}\NormalTok{ mapData, }
      \KeywordTok{aes}\NormalTok{(}\DataTypeTok{fill =}\NormalTok{ income_grp)) }\OperatorTok{+}
\StringTok{    }\CommentTok{# Externe Farbpalette, Beispiel viridis}
\StringTok{    }\CommentTok{# https://www.rdocumentation.org/packages/viridis/versions/0.5.1/topics/scale_color_viridis}
\StringTok{    }\NormalTok{viridis}\OperatorTok{::}\KeywordTok{scale_fill_viridis}\NormalTok{(}
      \CommentTok{# Diskrete Variable (Einkommensgruppen)}
      \DataTypeTok{discrete =} \OtherTok{TRUE}\NormalTok{,}
      \CommentTok{# Umkehr der Palette, damit dunkel = Niedrig}
      \DataTypeTok{direction =} \DecValTok{-1}\NormalTok{) }\OperatorTok{+}
\StringTok{    }\CommentTok{# ggrepel ist ein package, das Beschriftungen oder Labels so ausrichtet, dass es zu keinen Überlappungen kommmt}
\StringTok{    }\NormalTok{ggrepel}\OperatorTok{::}\KeywordTok{geom_label_repel}\NormalTok{(}
      \CommentTok{# man kann die ausgewählte Variable in ggplot vorab mit "subset" filtern}
      \DataTypeTok{data =} \KeywordTok{subset}\NormalTok{(mapData, income_grp }\OperatorTok{==}\StringTok{ "Niedrig"}\NormalTok{), }
      \CommentTok{# ohne stat = "sf_coordinates" kann ggrepel keine "geometry" Angaben verarbeiten}
      \DataTypeTok{stat =} \StringTok{"sf_coordinates"}\NormalTok{,}
      \KeywordTok{aes}\NormalTok{(}
        \DataTypeTok{geometry =}\NormalTok{ geometry,}
        \DataTypeTok{label =}\NormalTok{ name)) }\OperatorTok{+}
\StringTok{    }\CommentTok{# siehe theme settings oben}
\StringTok{    }\KeywordTok{theme_ggplot2_map_print_A4_1C}\NormalTok{() }\OperatorTok{+}
\StringTok{    }\CommentTok{# Beschriftungen}
\StringTok{    }\KeywordTok{labs}\NormalTok{(}
      \DataTypeTok{title =} \StringTok{"Titel"}\NormalTok{,}
      \DataTypeTok{subtitle =} \StringTok{"Untertitel"}\NormalTok{,}
      \DataTypeTok{caption =} \StringTok{"Fußnote"}\NormalTok{,}
      \DataTypeTok{tag =} \StringTok{"label"}\NormalTok{,}
      \DataTypeTok{fill =} \StringTok{"Titel Legende"}\NormalTok{) }\OperatorTok{+}
\StringTok{    }\KeywordTok{xlab}\NormalTok{(}\StringTok{"Beschriftung x"}\NormalTok{) }\OperatorTok{+}
\StringTok{    }\KeywordTok{ylab}\NormalTok{(}\StringTok{"Beschriftung y"}\NormalTok{) }
\end{Highlighting}
\end{Shaded}

\hypertarget{scatter-plot-mit-facet-wrap}{%
\paragraph{Scatter Plot mit Facet
Wrap}\label{scatter-plot-mit-facet-wrap}}

\begin{Shaded}
\begin{Highlighting}[]
\NormalTok{ScatterData <-}\StringTok{ }\NormalTok{data_WorldData }\OperatorTok\StringTok{ }
\StringTok{    }\KeywordTok{select}\NormalTok{(}
\NormalTok{    name,}
\NormalTok{    continent,}
\NormalTok{    inequality,}
\NormalTok{    well_being,}
\NormalTok{    gdp_cap_est,}
\NormalTok{    economy) }\OperatorTok\StringTok{ }
\StringTok{  }\KeywordTok{group_by}\NormalTok{(}
\NormalTok{    continent) }\OperatorTok\StringTok{ }
\StringTok{  }\KeywordTok{mutate}\NormalTok{(}\DataTypeTok{avg_gdp =} \KeywordTok{mean}\NormalTok{(gdp_cap_est, }\DataTypeTok{na.rm =} \OtherTok{TRUE}\NormalTok{)) }\OperatorTok\StringTok{ }
\StringTok{  }\KeywordTok{ungroup}\NormalTok{() }\OperatorTok\StringTok{ }
\StringTok{  }\KeywordTok{drop_na}\NormalTok{() }\OperatorTok\StringTok{ }
\StringTok{  }\KeywordTok{mutate}\NormalTok{(}
    \CommentTok{# Vereinigung der Kategorien}
    \DataTypeTok{economy =}\NormalTok{ forcats}\OperatorTok{::}\KeywordTok{fct_collapse}\NormalTok{(economy,}
      \DataTypeTok{entwickelt =} \KeywordTok{c}\NormalTok{(}
        \StringTok{"1. Developed region: G7"}\NormalTok{, }
        \StringTok{"2. Developed region: nonG7"}\NormalTok{),}
      \DataTypeTok{aufstrebend =} \KeywordTok{c}\NormalTok{(}
        \StringTok{"3. Emerging region: BRIC"}\NormalTok{, }
        \StringTok{"4. Emerging region: MIKT"}\NormalTok{, }
        \StringTok{"5. Emerging region: G20"}\NormalTok{),}
      \StringTok{"nicht-entwickelt"}\NormalTok{ =}\StringTok{ }\KeywordTok{c}\NormalTok{(}
        \StringTok{"6. Developing region"}\NormalTok{, }
        \StringTok{"7. Least developed region"}\NormalTok{))) }
  
\NormalTok{ScatterPlot <-}\StringTok{   }
\StringTok{  }\KeywordTok{ggplot}\NormalTok{(ScatterData) }\OperatorTok{+}
\StringTok{  }\KeywordTok{geom_point}\NormalTok{(}
    \KeywordTok{aes}\NormalTok{(}
\NormalTok{      inequality, }
\NormalTok{      well_being,}
    \CommentTok{# Anordung der Kontinente nach absteigender, durchschnittlicher Bevölkerung}
    \DataTypeTok{colour =}\NormalTok{ forcats}\OperatorTok{::}\KeywordTok{fct_reorder}\NormalTok{(continent, }\KeywordTok{desc}\NormalTok{(avg_gdp))),}
    \DataTypeTok{alpha =} \FloatTok{0.8}\NormalTok{) }\OperatorTok{+}\StringTok{ }
\StringTok{  }\KeywordTok{facet_wrap}\NormalTok{(}
    \OperatorTok{~}\StringTok{ }\NormalTok{economy, }
    \DataTypeTok{nrow =} \DecValTok{2}\NormalTok{) }\OperatorTok{+}
\StringTok{  }\KeywordTok{scale_colour_manual}\NormalTok{(}
    \DataTypeTok{values =}\NormalTok{ PAL_div_lowcontrast,}
    \DataTypeTok{guide =} \KeywordTok{guide_legend}\NormalTok{(}
                      \DataTypeTok{title.position =} \StringTok{"top"}\NormalTok{,}
                      \DataTypeTok{title=}\StringTok{"Kontinente"}\NormalTok{,}
                      \DataTypeTok{direction=}\StringTok{"horizontal"}\NormalTok{,}
                      \DataTypeTok{nrow =} \DecValTok{3}\NormalTok{,}
                      \DataTypeTok{ncol =} \DecValTok{2}\NormalTok{)) }\OperatorTok{+}
\StringTok{  }\KeywordTok{geom_smooth}\NormalTok{(}\KeywordTok{aes}\NormalTok{(}\DataTypeTok{x =}\NormalTok{ inequality, }\DataTypeTok{y =}\NormalTok{ well_being), }\DataTypeTok{method =} \StringTok{"lm"}\NormalTok{) }\OperatorTok{+}
\StringTok{  }\KeywordTok{theme_minimal}\NormalTok{() }\OperatorTok{+}
\StringTok{  }\KeywordTok{xlab}\NormalTok{(}\StringTok{"Wohlbefinden"}\NormalTok{) }\OperatorTok{+}
\StringTok{  }\KeywordTok{ylab}\NormalTok{(}\StringTok{"Ungleichheit"}\NormalTok{) }\OperatorTok{+}
\StringTok{  }\KeywordTok{theme}\NormalTok{(}
    \CommentTok{# Legenden Position, Alternativ: "top", "bottom", "right", "left"}
    \DataTypeTok{legend.position =} \KeywordTok{c}\NormalTok{(}\FloatTok{0.72}\NormalTok{, }\FloatTok{0.27}\NormalTok{),}
    \CommentTok{# Legenden Schrift fett}
    \DataTypeTok{legend.title =} \KeywordTok{element_text}\NormalTok{(}\DataTypeTok{face=}\StringTok{"bold"}\NormalTok{),}
    \CommentTok{# Abstand der Achsentitel zum Achsentext}
    \DataTypeTok{axis.title.x =} \KeywordTok{element_text}\NormalTok{(}\DataTypeTok{margin =} \KeywordTok{margin}\NormalTok{(}\DataTypeTok{t =} \DecValTok{15}\NormalTok{, }\DataTypeTok{r =} \DecValTok{0}\NormalTok{, }\DataTypeTok{b =} \DecValTok{0}\NormalTok{, }\DataTypeTok{l =} \DecValTok{0}\NormalTok{)),}
    \DataTypeTok{axis.title.y =} \KeywordTok{element_text}\NormalTok{(}\DataTypeTok{margin =} \KeywordTok{margin}\NormalTok{(}\DataTypeTok{t =} \DecValTok{0}\NormalTok{, }\DataTypeTok{r =} \DecValTok{15}\NormalTok{, }\DataTypeTok{b =} \DecValTok{0}\NormalTok{, }\DataTypeTok{l =} \DecValTok{0}\NormalTok{)))}
\end{Highlighting}
\end{Shaded}

\hypertarget{population-pyramid}{%
\paragraph{Population Pyramid}\label{population-pyramid}}

\hypertarget{bar-charts}{%
\paragraph{Bar Charts}\label{bar-charts}}

\begin{Shaded}
\begin{Highlighting}[]
\NormalTok{countryList <-}\StringTok{ }\KeywordTok{c}\NormalTok{(}\StringTok{"Germany"}\NormalTok{,}\StringTok{"Romania"}\NormalTok{,}\StringTok{"Bulgaria"}\NormalTok{,}\StringTok{"Syrian Arab Republic"}\NormalTok{)}

\NormalTok{BarData <-}\StringTok{ }\NormalTok{data_popFM_long }\OperatorTok\StringTok{ }
\StringTok{  }\CommentTok{# filtert Kontinente und Länder Gruppen heraus}
\StringTok{  }\KeywordTok{filter}\NormalTok{(}\OperatorTok{!}\NormalTok{country_code }\OperatorTok{>=}\StringTok{ }\DecValTok{900}\NormalTok{) }\OperatorTok\StringTok{ }
\StringTok{  }\NormalTok{dplyr}\OperatorTok{::}\KeywordTok{group_by}\NormalTok{(}
\NormalTok{    year,}
\NormalTok{    country,}
\NormalTok{    gender) }\OperatorTok\StringTok{ }
\StringTok{  }\NormalTok{dplyr}\OperatorTok{::}\KeywordTok{mutate}\NormalTok{(}
    \DataTypeTok{percPop_country_gender =} \KeywordTok{round}\NormalTok{(population}\OperatorTok{/}\KeywordTok{sum}\NormalTok{(population)}\OperatorTok{*}\DecValTok{100}\NormalTok{,}\DecValTok{1}\NormalTok{)) }\OperatorTok\StringTok{ }
\StringTok{  }\KeywordTok{filter}\NormalTok{(year }\OperatorTok{==}\StringTok{ "2015"}\NormalTok{) }\OperatorTok\StringTok{ }
\StringTok{  }\KeywordTok{filter}\NormalTok{(country }\OperatorTok\StringTok{ }\NormalTok{countryList)}
\NormalTok{t <-}\StringTok{ }\NormalTok{data_popFM_long }\OperatorTok\StringTok{ }\KeywordTok{select}\NormalTok{(country) }\OperatorTok\StringTok{ }\KeywordTok{distinct}\NormalTok{()}
\NormalTok{plotBarGrouped <-}\StringTok{ }\NormalTok{BarData }\OperatorTok\StringTok{ }
\StringTok{    }\KeywordTok{ggplot}\NormalTok{( }
      \KeywordTok{aes}\NormalTok{(}
        \DataTypeTok{x=}\NormalTok{age, }
        \DataTypeTok{y=}\NormalTok{percPop_country_gender, }
        \DataTypeTok{group=}\NormalTok{gender, }
        \DataTypeTok{fill=}\NormalTok{gender)) }\OperatorTok{+}
\StringTok{    }\KeywordTok{geom_bar}\NormalTok{(}
      \DataTypeTok{position =} \StringTok{"dodge"}\NormalTok{, }\DataTypeTok{width=}\FloatTok{0.7}\NormalTok{,}
      \DataTypeTok{stat =} \StringTok{"identity"}\NormalTok{) }\OperatorTok{+}
\StringTok{    }\KeywordTok{scale_fill_manual}\NormalTok{(}\DataTypeTok{labels =} \KeywordTok{c}\NormalTok{(}\StringTok{"weiblich"}\NormalTok{, }\StringTok{"männlich"}\NormalTok{), }\DataTypeTok{values =} \KeywordTok{c}\NormalTok{(}\StringTok{"blue"}\NormalTok{, }\StringTok{"red"}\NormalTok{)) }\OperatorTok{+}
\StringTok{    }\KeywordTok{scale_x_discrete}\NormalTok{(}\DataTypeTok{guide =} \KeywordTok{guide_axis}\NormalTok{(}\DataTypeTok{angle =} \DecValTok{90}\NormalTok{)) }\OperatorTok{+}
\StringTok{    }\KeywordTok{labs}\NormalTok{(}
      \DataTypeTok{title =} \StringTok{"Anteile der Bevölkerung nach Geschlecht und Altersgruppe"}\NormalTok{,}
      \DataTypeTok{subtitle =} \StringTok{"in West-europäischen Ländern, 2015"}\NormalTok{,}
      \DataTypeTok{caption =} \StringTok{"Datenset wpp2015, Age- and sex-specific populationestimates and projections "}\NormalTok{,}
      \DataTypeTok{tag =} \StringTok{"Abb. 01"}\NormalTok{,}
      \DataTypeTok{fill =} \StringTok{"Geschlecht"}\NormalTok{) }\OperatorTok{+}
\StringTok{    }\KeywordTok{xlab}\NormalTok{(}\StringTok{"Altersgruppen"}\NormalTok{) }\OperatorTok{+}
\StringTok{    }\KeywordTok{ylab}\NormalTok{(}\StringTok{"Anteil an geschlechtsspezifischer Hauptwohnbevölkerung"}\NormalTok{) }\OperatorTok{+}
\StringTok{  }\KeywordTok{facet_wrap}\NormalTok{(}\OperatorTok{~}\StringTok{ }\NormalTok{country) }\OperatorTok{+}
\StringTok{  }\KeywordTok{theme_minimal}\NormalTok{()}



\NormalTok{plotBarFill <-}\StringTok{ }\NormalTok{BarData }\OperatorTok\StringTok{ }
\StringTok{    }\KeywordTok{ggplot}\NormalTok{() }\OperatorTok{+}
\StringTok{    }\KeywordTok{geom_bar}\NormalTok{(}
      \DataTypeTok{data =} \KeywordTok{subset}\NormalTok{(BarData, gender }\OperatorTok{==}\StringTok{ "F"}\NormalTok{), }
      \KeywordTok{aes}\NormalTok{(}
        \DataTypeTok{x=}\NormalTok{age, }
        \DataTypeTok{y=}\NormalTok{percPop_country_gender),}
      \DataTypeTok{fill =} \OtherTok{NA}\NormalTok{,}
      \DataTypeTok{color =} \StringTok{"Black"}\NormalTok{,}
      \CommentTok{# position = "identity",}
      \DataTypeTok{stat=}\StringTok{"identity"}\NormalTok{) }\OperatorTok{+}
\StringTok{    }\KeywordTok{geom_bar}\NormalTok{(}
      \DataTypeTok{data =} \KeywordTok{subset}\NormalTok{(BarData, gender }\OperatorTok{==}\StringTok{ "M"}\NormalTok{), }
      \KeywordTok{aes}\NormalTok{(}
        \DataTypeTok{x=}\NormalTok{age, }
        \DataTypeTok{y=}\NormalTok{percPop_country_gender),}
      \DataTypeTok{alpha =} \FloatTok{0.3}\NormalTok{,}
      \CommentTok{# position = "identity",}
      \DataTypeTok{stat=}\StringTok{"identity"}\NormalTok{) }\OperatorTok{+}
\StringTok{    }\KeywordTok{scale_x_discrete}\NormalTok{(}\DataTypeTok{guide =} \KeywordTok{guide_axis}\NormalTok{(}\DataTypeTok{angle =} \DecValTok{90}\NormalTok{)) }\OperatorTok{+}
\StringTok{    }\CommentTok{# scale_colour_manual(labels = c("männlich", "weiblich"), values=gender, aesthetics = c("colour", "fill")) +}
\StringTok{    }\KeywordTok{scale_colour_manual}\NormalTok{(}\DataTypeTok{labels =} \KeywordTok{c}\NormalTok{(}\StringTok{"männlich"}\NormalTok{, }\StringTok{"weiblich"}\NormalTok{), }\DataTypeTok{values=}\KeywordTok{c}\NormalTok{(}\StringTok{"lightblue4"}\NormalTok{, }\StringTok{"red"}\NormalTok{)) }\OperatorTok{+}
\StringTok{    }\KeywordTok{labs}\NormalTok{(}
      \DataTypeTok{title =} \StringTok{"Anteile der Bevölkerung nach Geschlecht und Altersgruppe"}\NormalTok{,}
      \DataTypeTok{subtitle =} \StringTok{"in West-europäischen Ländern, 2015"}\NormalTok{,}
      \DataTypeTok{caption =} \StringTok{"Datenset wpp2015, Age- and sex-specific populationestimates and projections "}\NormalTok{,}
      \DataTypeTok{tag =} \StringTok{"Abb. 01"}\NormalTok{,}
      \DataTypeTok{fill =} \StringTok{"Geschlecht"}\NormalTok{) }\OperatorTok{+}
\StringTok{    }\KeywordTok{xlab}\NormalTok{(}\StringTok{"Altersgruppen"}\NormalTok{) }\OperatorTok{+}
\StringTok{    }\KeywordTok{ylab}\NormalTok{(}\StringTok{"Anteil an geschlechtsspezifischer Hauptwohnbevölkerung"}\NormalTok{) }\OperatorTok{+}
\StringTok{  }\KeywordTok{facet_wrap}\NormalTok{(}\OperatorTok{~}\StringTok{ }\NormalTok{country, }\DataTypeTok{nrow =} \DecValTok{2}\NormalTok{, }\DataTypeTok{ncol =} \DecValTok{2}\NormalTok{) }\OperatorTok{+}
\StringTok{  }\KeywordTok{theme_minimal}\NormalTok{()}


\NormalTok{plotBarOverlay <-}\StringTok{ }\NormalTok{BarData }\OperatorTok\StringTok{ }
\StringTok{    }\KeywordTok{ggplot}\NormalTok{(      }
      \KeywordTok{aes}\NormalTok{(}
        \DataTypeTok{x=}\NormalTok{age, }
        \DataTypeTok{y=}\NormalTok{percPop_country_gender,}
        \DataTypeTok{fill =}\NormalTok{ gender}
        \CommentTok{# color = gender,}
        \CommentTok{# alpha = gender}
\NormalTok{        )) }\OperatorTok{+}
\StringTok{    }\KeywordTok{geom_bar}\NormalTok{(}
      \DataTypeTok{alpha =} \FloatTok{0.6}\NormalTok{,}
      \DataTypeTok{position =} \StringTok{"identity"}\NormalTok{,}
      \DataTypeTok{stat=}\StringTok{"identity"}\NormalTok{) }\OperatorTok{+}
\StringTok{  }\CommentTok{# scale_colour_manual(values=c("lightblue4", "red")) +}
\StringTok{  }\KeywordTok{scale_fill_manual}\NormalTok{(}
    \DataTypeTok{labels =} \KeywordTok{c}\NormalTok{(}\StringTok{"männlich"}\NormalTok{, }\StringTok{"weiblich"}\NormalTok{),}
    \DataTypeTok{values=}\KeywordTok{c}\NormalTok{(}\StringTok{"lightblue"}\NormalTok{, }\StringTok{"pink"}\NormalTok{)) }\OperatorTok{+}
\StringTok{  }\KeywordTok{scale_x_discrete}\NormalTok{(}\DataTypeTok{guide =} \KeywordTok{guide_axis}\NormalTok{(}\DataTypeTok{angle =} \DecValTok{90}\NormalTok{)) }\OperatorTok{+}
\StringTok{  }\CommentTok{# scale_alpha_manual(values=c(.3, .8)) +}
\StringTok{    }\KeywordTok{labs}\NormalTok{(}
      \DataTypeTok{title =} \StringTok{"Anteile der Bevölkerung nach Geschlecht und Altersgruppe"}\NormalTok{,}
      \DataTypeTok{subtitle =} \StringTok{"in West-europäischen Ländern, 2015"}\NormalTok{,}
      \DataTypeTok{caption =} \StringTok{"Datenset wpp2015, Age- and sex-specific populationestimates and projections "}\NormalTok{,}
      \DataTypeTok{tag =} \StringTok{"Abb. 01"}\NormalTok{,}
      \DataTypeTok{fill =} \StringTok{"Geschlecht"}\NormalTok{) }\OperatorTok{+}
\StringTok{    }\KeywordTok{xlab}\NormalTok{(}\StringTok{"Altersgruppen"}\NormalTok{) }\OperatorTok{+}
\StringTok{    }\KeywordTok{ylab}\NormalTok{(}\StringTok{"Anteil an geschlechtsspezifischer Hauptwohnbevölkerung"}\NormalTok{) }\OperatorTok{+}
\StringTok{  }\KeywordTok{facet_wrap}\NormalTok{(}\OperatorTok{~}\StringTok{ }\NormalTok{country, }\DataTypeTok{nrow =} \DecValTok{2}\NormalTok{, }\DataTypeTok{ncol =} \DecValTok{2}\NormalTok{) }\OperatorTok{+}
\StringTok{  }\KeywordTok{theme_minimal}\NormalTok{() }\OperatorTok{+}
\StringTok{  }\KeywordTok{theme}\NormalTok{(}\DataTypeTok{legend.position =} \StringTok{"top"}\NormalTok{)}
\end{Highlighting}
\end{Shaded}

\newpage

\hypertarget{plot---layout}{%
\section{Plot - Layout}\label{plot---layout}}

\hypertarget{gemischtes-1-und-2-spalten-layout}{%
\subsection{Gemischtes 1 und 2 Spalten
Layout}\label{gemischtes-1-und-2-spalten-layout}}

\begin {multicols}{2}

\includegraphics{ggplot2_files/figure-latex/unnamed-chunk-7-1.pdf}

Lorem ipsum dolor sit amet, consetetur sadipscing elitr, sed diam nonumy
eirmod tempor invidunt ut labore et dolore magna aliquyam erat, sed diam
voluptua. At vero eos et accusam et justo duo dolores et ea rebum.

\columnbreak

\includegraphics{ggplot2_files/figure-latex/unnamed-chunk-8-1.pdf}

Lorem ipsum dolor sit amet, consetetur sadipscing elitr, sed diam nonumy
eirmod tempor invidunt ut labore et dolore magna aliquyam erat, sed diam
voluptua. At vero eos et accusam et justo duo dolores et ea rebum.

\end {multicols}

\includegraphics{ggplot2_files/figure-latex/unnamed-chunk-9-1.pdf}

\includegraphics{ggplot2_files/figure-latex/unnamed-chunk-10-1.pdf}

\begin{Shaded}
\begin{Highlighting}[]
\NormalTok{plotBarOverlay}
\end{Highlighting}
\end{Shaded}

\includegraphics{ggplot2_files/figure-latex/unnamed-chunk-11-1.pdf}

\begin{Shaded}
\begin{Highlighting}[]
\NormalTok{plotBarFill}
\end{Highlighting}
\end{Shaded}

\includegraphics{ggplot2_files/figure-latex/unnamed-chunk-12-1.pdf}

\end{document}
