% Stadt Dortmund, Dortmunder Statistik
% Autor: Fabian Koch, fkoch@stadtdo.de


% Dies ist eine Vorlage für eine RMardkown Präambel, ausgerichtet auf spezifische Kriterien von Printprodukten der Dortmunder Statistik
% Sie bildet nur ein Bruchteil der verfügbaren Optionen ab

% Kurzeinführung mit weiterführenden Links: https://bookdown.org/yihui/rmarkdown-cookbook/latex-template.html

% Weiterführende Ressourcen:
% Pandoc Template Anleitung mit allen Optionen: https://pandoc.org/MANUAL.html#templates

% Zur Benutzung des Templates muss diese im YAML Header des RMarkdown Dokumentes integriert werden:
% Beispiel:
% output:
%   bookdown::pdf_document2:
%     includes:
%       in_header: preamble.tex

% Hinweise:
% Die Präambel ist in KATEGORIEN unterteilt
% Kommentare im Skript erfolgen jeweils über der jeweiligen Variable



%%%%% SCHRIFT %%%%%
\usepackage[utf8]{inputenc}
\usepackage{fontspec}
% hier ein Beispiel für den Import der Dortmunder CD Schrift "frutiger"
% es wird entsprechend der Dateiname im Windows font Verzeichnis angegeben
\setmainfont{LTe50331_0.ttf}[
    BoldFont = LTe50333_0.ttf ,
    ItalicFont = LTe50332_0.ttf ,
    BoldItalicFont = LTe50332_0.ttf]

% Hinweis:
% das R package {{systemfonts}} zeigt installierte Schriften an
% im Markdown Dokument müssen dann die Schriften geladen werden
% im Markdown Dokument müssen dann die Schriften geladen werden
% windowsFonts(sans="Frutiger LT 55 Roman")
% loadfonts(device="win")
% loadfonts(device="postscript")
% dies geschieht im gelieferten Beispiel im Skript "fonts_colors" im Unterordner scripts


%%%%% Seiten Layout %%%%%
% das package {multicol} erlaubt mehrspaltige Seiten Layouts
\usepackage{multicol}
% Work around für pandoc markdown
% https://stackoverflow.com/questions/40982836/latex-multicolumn-block-in-pandoc-markdown
% \newcommand{\hideFromPandoc}[1]{#1}
% \hideFromPandoc{
%       \let\Begin\begin
%       \let\End\end
%       }
% Spaltenabstand
\setlength{\columnsep}{15pt}



%%%%% FARBEN %%%%%
% Farbdefinitionen zur weiteren Verwendung
\definecolor{DoStat}{RGB}{0, 84, 189} 
\definecolor{DoGray}{RGB}{33, 36, 39}
% Definition  des Befehls "boldDoStat", um gleichzeitig fett und eingefärbte Schrift zu ermöglichen
\newcommand\boldDoStat[1]{\textcolor{DoStat}{\textbf{#1}}} 
% Tabellen Rahmen Farbe
\usepackage{colortbl}
\arrayrulecolor{DoGray}


%%%%% KOPF-/FUßZEILEN
\usepackage{fancyhdr}
\pagestyle{fancy}
\fancyhf{}
\fancyhead[LE]{\color{DoStat}\textbf{\nouppercase\leftmark}}
\fancyhead[RO]{\color{DoStat}\textbf{\nouppercase\leftmark}}
\fancyfoot[LE]{\color{DoStat} \textbf{\thepage} \hspace{0.5cm} stadt dortmund • Name der Veröffentlichung • Jahr}
\fancyfoot[RO]{\color{DoStat} stadt dortmund • Name der Veröffentlichung • Jahr \hspace{0.5cm} \textbf{\thepage}}
\setlength{\headheight}{12.8pt}
\renewcommand{\headrulewidth}{0pt}


%%%%% BESCHRIFTUNGEN %%%%%
\usepackage[justification=RaggedRight,font=small,singlelinecheck=false]{caption}
\usepackage{color}
\DeclareCaptionFont{blue}{\color{DoStat}}
\captionsetup{labelfont={blue,footnotesize}}
\captionsetup[table]{belowskip=4pt}
\captionsetup[figure]{belowskip=4pt}
\captionsetup[table]{aboveskip=4pt}
\captionsetup[figure]{aboveskip=4pt}

% Positioniert Beschriftungen über Tabelle
\usepackage{floatrow}
\floatsetup[figure]{capposition=top}
\floatsetup[table]{capposition=top}


%%%%% FLOATS %%%%%
% Das package Float erlaubt das freie Positionieren von Grafiken und Tabellen. 
% insb. in Verbindung mit einem 2-spaltigem Layout und RMarkdown Beschriftungen mit fig.cap= "Beschriftung", werden  Grafiken "geschluckt". Das Umgehen von Floats ist ebenfalls wichtig, wenn 2-Grafiken zusammen auf eine Seite "gezwungen" werden sollen.
% im RMarkdown Chunk müssen dann folgende Parameter gesetzt werden:
% fig.cap= "Beschriftung", fig.pos="H"
% https://bookdown.org/yihui/rmarkdown-cookbook/figure-placement.html
% https://stackoverflow.com/questions/16626462/figure-position-in-markdown-when-converting-to-pdf-with-knitr-and-pandoc
\usepackage{float}


%%%%% SONSTIGES %%%%%
% erlaubt Einbindung von Grafiken über \includegraphics{}
\usepackage{graphicx} 


% Tabellen
\usepackage{tabularx}

\usepackage{makecell}
\renewcommand\cellgape{\Gape[4pt]}


% Falls Seiten im Querformat gewünscht sind
\usepackage{lscape}
\newcommand{\blandscape}{\begin{landscape}}
\newcommand{\elandscape}{\end{landscape}}


% keine Titelseite, da diese in Dortmund extern entwickelt und angebunden wird
\AtBeginDocument{\let\maketitle\relax} 


\geometry{left=2.3cm,right=2.3cm,top=2.3cm,bottom=2.3cm}


% deutsche Bezeichnung für Verzeichnisse und Beschriftungen
\renewcommand{\figurename}{Abb.}
\renewcommand{\tablename}{Tab.}
\renewcommand{\contentsname}{Inhaltsverzeichnis}
\renewcommand{\listfigurename}{Abbildungsverzeichnis}
\renewcommand{\listtablename}{Tabellenverzeichnis}

% \renewcommand{\topfraction}{.85}
% \renewcommand{\bottomfraction}{.7}
% \renewcommand{\textfraction}{.15}
% \renewcommand{\floatpagefraction}{.66}
% \setcounter{topnumber}{3}
% \setcounter{bottomnumber}{3}
% \setcounter{totalnumber}{4}

